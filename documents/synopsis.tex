\documentclass[a4paper, 10pt]{article}
\usepackage[danish]{babel}
\usepackage[T1]{fontenc}
\usepackage[utf8]{inputenc}
\usepackage{fancyhdr} % hoved- og sidefod
\usepackage{amsfonts}
\usepackage{listings}
\usepackage{amssymb,amsmath}
\usepackage{graphicx}
\pagestyle{fancy} 
\lhead{Bachelorprojekt\\ Synopsis}
\rhead{Taus Møller\\ Morten Rasmussen}
\lstset{ 
language=Python,                % choose the language of the code
basicstyle=\footnotesize,       % the size of the fonts that are used for the code
numbers=left,                   % where to put the line-numbers
numberstyle=\footnotesize,      % the size of the fonts that are used for the line-numbers
stepnumber=1,                   % the step between two line-numbers. If it's 1 each line will be numbered
numbersep=5pt,                  % how far the line-numbers are from the code
showspaces=false,               % show spaces adding particular underscores
showstringspaces=false,         % underline spaces within strings
showtabs=false,                 % show tabs within strings adding particular underscores
%frame=single,	                % adds a frame around the code
tabsize=4,	                % sets default tabsize to 2 spaces
captionpos=b,                   % sets the caption-position to bottom
breaklines=true,                % sets automatic line breaking
breakatwhitespace=false,        % sets if automatic breaks should only happen at whitespace
title=\lstname,                 % show the filename of files included with \lstinputlisting; also try caption instead of title
%escapeinside={\%*}{*)}          % if you want to add a comment within your code
}


\begin{document}
\section*{Synopsis til Bachelor projekt 2010}
\subsection*{Title}
Procedural content creation for benchmarking physics simulation.
\subsection*{Problem Formulation}
Is it possible to extend the existing prototype for benchmarking physics engines, to include procedural content
creation.
Is it possible to create viable data structures to contain 3D objects for use in dynamics simulations, including relevant physical attributes. The aim of the
project is is to develop a way to create objects inside the prototype using procedural modelling without the use of external tool.

\subsection*{Topic delimitations}
\begin{itemize}
	\item The project will be limited to primitives e.g. cubes and spheres and there combinations. 
	\item It will only be possible to create and place objects by inputting parameters and not by dragging and
	dropping.
	\item The objects will be presumed to be made of a homogeneous material to easier calculate center of gravity
\end{itemize}
\subsection*{Motivation}
It can be cumbersome to test physics engines when time has to be wasted
setting up test scenarios in standard tools. If its possible to use procedural modelling to make test data it will
improve the workflow for benchmarking physic engines. It will make it easier to set up different test environments and recycle the scenarios with
different configurations. 

These scenarios will be made out of primitives and combinations of primitives, so an obvious criteria for the project is to make it 
easy to setup and change large sets of primitives.
With procedural content creation it is possible to create completely different test cases with only a few minor changes
to the input. With procedural content creation it is easy to create large test scenarios that will test the limits of the
physics engine. 

With further development these data structures will possibly be able to make it easier to create more complex scenarios
and a better benchmarking tool. These complex scenarios will contribute to a more exhaustive and complete test.
\subsection*{Evaluation}
To complete our goal we will need to write a data structure that can hold relevant attributes for a 3D object and can be
created procedurally. We need to design a way with which the objects can be created within the benchmarking tool.
\subsection*{Project schedule}
\begin{description}
\item[Friday 24. September: Hand in of synopsis]
\item[Week 39 to 41: Research of physics and 3D engine] We need to do some general research in order to get a better
understanding for the subject of physics simulation. Among other things we need to find out what information is relevant to keep in the
data structure.
\item[Week 39 to 42: Code digging in existing prototype]
We need to look through the code of the existing prototype to get to know it and find out how it is put together. This
is needed because we need to know where our project fits in and how it is best integrated in the prototype.
\item[Week 42 to 45: Analysis of problem and its solution]
After gathering information we need to decide how we best solve the problem.
\item[Week 45 to 51: Extending prototype with procedural content creation] This is where we implement our solution into
the existing prototype.
\item[Week 48 to 1: Writing report]
This is the final part of writing the report. We will be writing during the entire process but in these weeks we will be
focusing more on the report
\item[Week 1 to 2:  Proofreading]
By this time we should be done with the report and be making the final changes and the last proofreading.
\item[Tuesday 11. January: Hand in of report]
\item[Wednesday 12. January to 27. January: Preparation for defence]
\item[27-28 January: Defence of project]
\end{description}
\end{document}
